\documentclass[12pt, a4paper]{article}
\usepackage[english]{babel}
\usepackage{graphicx}
\usepackage{url}
\renewcommand{\title}{On the use of knowledge-based media to enhance the social life of elderly}
\newcommand{\authorA}{Harrie Oosterhuis}
\newcommand{\studnummerA}{10196129}
\newcommand{\authorB}{Dasyel Willems}
\newcommand{\studnummerB}{10172548}
\newcommand{\authorC}{Carla Groenland}
\newcommand{\studnummerC}{10208429}
\newcommand{\authorD}{Fabian Voorter}
\newcommand{\studnummerD}{10218807}
\newcommand{\authorE}{Jelle van Assema}
\newcommand{\studnummerE}{10200363}
\newcommand{\einddatum}{\today}
\newcommand{\content}{Report Kennisgebaseerde media}

\usepackage[pdftex,pagebackref,colorlinks,linkcolor=black,urlcolor=blue,citecolor=black]{hyperref}
\begin{document}

\begin{titlepage}
\vspace*{10pt}
\begin{center}
{\LARGE \textbf{\bf \title}\par}\bigskip
{\Large \authorA, \authorB, \authorC, \authorD, \authorE
\par}\bigskip
{\Large \einddatum\par}\bigskip\bigskip
\vspace{\stretch{1}}
\vspace{\stretch{1}}
Bachelor Kunstmatige Intelligentie \par\smallskip
Faculteit der Natuurwetenschappen, Wiskunde en Informatica\par\smallskip
Universiteit van Amsterdam\par\bigskip
\includegraphics[width=0.075\hsize]{uvalogo}
\end{center}
\end{titlepage}

\begin{titlepage}
\noindent
\begin{minipage}{0.85\hsize}
\section*{Abstract}
This report describes the possibilities of using knowledge-based media systems to enhance the social life of elderly. 
\end{minipage}

\vfill

\noindent Title: \title\\
Authors: \authorA, \studnummerA\\
\hphantom{Authors: }\authorB, , \studnummerB\\
\hphantom{Authors:\ }\authorC,  \studnummerC\\
\hphantom{Authors:\ }\authorD,  \studnummerD\\
\hphantom{Authors:\ }\authorE, \studnummerE\\
Date: \einddatum

\bigskip\noindent
Faculteit der Natuurwetenschappen, Wiskunde en Informatica\\
Universiteit van Amsterdam\\
Science Park 904, 1098 XH Amsterdam\\
\url{http://www.science.uva.nl/home.cfm}
\end{titlepage}

\tableofcontents 
\newpage
\section{Introduction}

Life in an old peoples home can be quite limiting, the inhabitants have often lost a lot of their mobility. Frequently they move into the home without getting to know their new neighbours, having their social interaction limited to the visits of their family. In this report, we discuss a system which tries to enhance the life of the elderly by introducing them to an advanced system that provides them with entertainment and supports social interaction. The system is designed to respect the privacy and much needed rest of the elderly and is equipped with a user-friendly interface.

The application of the system places three interactive screens in each residence. A large screen replacing their television, a light tablet that can be held as a magazine and a coffee table with an interactive screen as its table-top. The screens can display media from both online and traditional sources. The system determines autonomously what the users enjoys the most and uses this knowledge to gather appropriate content. By doing so users that are unfamiliar with online sources can enjoy their content regardless. The provided content can range from television programmes or documentaries available online to family photos shared on social media.

Furthermore, the system enhances the social life of the elderly by planning social activities based on their personal preferences. Activities already popular among the elderly like trivia and card games can be suggested by our system. Having knowledge of the preferences of other residents, the system can match neighbours with the activity. The user gets the option to invite these neighbours to their home for an activity. Residents will not only be matched on their preferred activities but also on their interests. The aforementioned interactive screens react to the residents in the room, suggesting content matching their interests. For instance photos of family or important places and events may be displayed. An important place may be an old school or other place of nostalgic value. When a photo is interacted with more content regarding the same subject is displayed. By doing so conversation topics can be introduced, easing social interaction between neighbours.

Considering the inconformity of the elderly with present-day technology the user interface is both intuitive and simple. Buttons will be large and keyboard input will be limited. Using a knowledge-based approach media is sorted and provided in categories that make sense to its user. For instance the user has the option to ask the system to look for content similar to what just has been watched. Resulting in simple interaction that enables intuitive interaction.

\section{Goal and system tasks}

\section{Literature review}

\section{System description}

\section{Discussion of approach}

\section{Conclusion}
\begin{thebibliography}{x}
\addcontentsline{toc}{section}{Bibliografy} 

\end{thebibliography}

\end{document}
