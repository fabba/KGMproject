\documentclass[12pt, a4paper]{article}
\usepackage[english]{babel}
\usepackage{graphicx}
\usepackage{url}
\renewcommand{\title}{On the use of knowledge-based media to enhance the social life of elderly}
\newcommand{\authorA}{Harrie Oosterhuis}
\newcommand{\studnummerA}{10196129}
\newcommand{\authorB}{Dasyel Willems}
\newcommand{\studnummerB}{10172548}
\newcommand{\authorC}{Carla Groenland}
\newcommand{\studnummerC}{10208429}
\newcommand{\authorD}{Fabian Voorter}
\newcommand{\studnummerD}{10218807}
\newcommand{\authorE}{Jelle van Assema}
\newcommand{\studnummerE}{10200363}
\newcommand{\einddatum}{\today}
\newcommand{\content}{Report Kennisgebaseerde media}

\usepackage[pdftex,pagebackref,colorlinks,linkcolor=black,urlcolor=blue,citecolor=black]{hyperref}
\begin{document}

\begin{titlepage}
\vspace*{10pt}
\begin{center}
{\LARGE \textbf{\bf \title}\par}\bigskip
{\Large \authorA, \authorB, \authorC, \authorD, \authorE
\par}\bigskip
{\Large \einddatum\par}\bigskip\bigskip
\vspace{\stretch{1}}
\vspace{\stretch{1}}
Bachelor Kunstmatige Intelligentie \par\smallskip
Faculteit der Natuurwetenschappen, Wiskunde en Informatica\par\smallskip
Universiteit van Amsterdam\par\bigskip
\includegraphics[width=0.075\hsize]{uvalogo}
\end{center}
\end{titlepage}

\begin{titlepage}
\noindent
\begin{minipage}{0.85\hsize}
\section*{Abstract}
This report describes the possibilities of using knowledge-based media systems to enhance the social life of elderly. 
\end{minipage}

\vfill

\noindent Title: \title\\
Authors: \authorA, \studnummerA\\
\hphantom{Authors: }\authorB, , \studnummerB\\
\hphantom{Authors:\ }\authorC,  \studnummerC\\
\hphantom{Authors:\ }\authorD,  \studnummerD\\
\hphantom{Authors:\ }\authorE, \studnummerE\\
Date: \einddatum

\bigskip\noindent
Faculteit der Natuurwetenschappen, Wiskunde en Informatica\\
Universiteit van Amsterdam\\
Science Park 904, 1098 XH Amsterdam\\
\url{http://www.science.uva.nl/home.cfm}
\end{titlepage}

\tableofcontents 
\newpage
\section{Introduction}
<<<<<<< HEAD


Life in an old peoples home can be quite limiting, the inhabitants have often lost a lot of their mobility. Often they move into the home without knowing their neighbours, having their social interaction limited to the visits of their family. In this report, we discuss a system which tries to enhance the life of elderly by introducing them to an advanced system that provides them with entertainment and supports social interaction. The system is designed to respect the privacy and much needed rest of the elderly and is equipped with a user-friendly interface.

The system reduces the boredom en loneliness of the elderly by providing them with user-based entertainment. [EXPLANATION OF MEDIA] Each room

=======
Life in an old peoples home can be quite limiting, the inhabitants have often lost a lot of their mobility. Often they move into the home without knowing their neighbours, having their social interaction limited to the visits of their family. In this report, we discuss a system which tries to enhance the life of elderly by introducing them to an advanced system that provides them with entertainment and supports social interaction. The system is designed to respect the privacy and much needed rest of the elderly and is equipped with a user-friendly interface.

The application of the system 
The system reduces the boredom and loneliness of the elderly by providing them with user-based entertainment. [EXPLANATION OF MEDIA] Each room
>>>>>>> 7b1cb7b41961e560a503fa352ccfa07b31aff7ea

Furthermore, the system enhances the social life of the elderly by planning several social activities for the elderly based on their personal needs. Activities like trivia, bingo, domino, rummikub and some card games can be played by the elderly. Activities which do not require any speed will be available for the elderly. Also to enhance the social life of the elderly, activities which require mulitple person are the priority.

The communication of the elderly is through [EXPLANATION INTERFACE]

The system adapts to every user by gathering a lot of data about the person, both about his life history, interests and family and about his interactions with the system. As an example, the system tries to recognize patterns in which type of entertainment a person in most pleased with and on which occasions. Through large amounts of data, the system gets to know each user and behaves appropriately.

\section{Goal and system tasks}

\section{Literature review}

\section{System description}

\section{Discussion of approach}

\section{Conclusion}
\begin{thebibliography}{x}
\addcontentsline{toc}{section}{Bibliografy} 

\end{thebibliography}

\end{document}